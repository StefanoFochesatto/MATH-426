%%%%%%%%%%%%%%%%%%%%%%%%%%%%%%%%%%%%%%%%%%%%%%%%%%%%%%%%%%%%%%%%%%%%%%%%%%%%%%%%%%%%%%%
%%%%%%%%%%%%%%%%%%%%%%%%%%%%%%%%%%%%%%%%%%%%%%%%%%%%%%%%%%%%%%%%%%%%%%%%%%%%%%%%%%%%%%%
% 
% This top part of the document is called the 'preamble'.  Modify it with caution!
%
% The real document starts below where it says 'The main document starts here'.

\documentclass[12pt]{article}

\usepackage{amssymb,amsmath,amsthm}
\usepackage[top=1in, bottom=1in, left=1.25in, right=1.25in]{geometry}
\usepackage{fancyhdr}
\usepackage{graphicx}
\usepackage{enumerate}
\usepackage{verbatim}
\usepackage{listings}
\usepackage{float}


% Comment the following line to use TeX's default font of Computer Modern.
\usepackage{times,txfonts}

\newtheoremstyle{homework}% name of the style to be used
  {18pt}% measure of space to leave above the theorem. E.g.: 3pt
  {12pt}% measure of space to leave below the theorem. E.g.: 3pt
  {}% name of font to use in the body of the theorem
  {}% measure of space to indent
  {\bfseries}% name of head font
  {:}% punctuation between head and body
  {2ex}% space after theorem head; " " = normal interword space
  {}% Manually specify head
\theoremstyle{homework} 

% Set up an Exercise environment and a Solution label.
\newtheorem*{exercisecore}{\@currentlabel}
\newenvironment{exercise}[1]
{\def\@currentlabel{#1}\exercisecore}
{\endexercisecore}

\newcommand{\localhead}[1]{\par\smallskip\noindent\textbf{#1}\nobreak\\}%
\newcommand\solution{\localhead{Solution:}}



% \newcommand{includematlab}[1]{\verbatiminput{#1}}

%%%%%%%%%%%%%%%%%%%%%%%%%%%%%%%%%%%%%%%%%%%%%%%%%%%%%%%%%%%%%%%%%%%%%%%%
%
% Stuff for getting the name/document date/title across the header
\makeatletter
\RequirePackage{fancyhdr}
\pagestyle{fancy}
\fancyfoot[C]{\ifnum \value{page} > 1\relax\thepage\fi}
\fancyhead[L]{\ifx\@doclabel\@empty\else\@doclabel\fi}
\fancyhead[C]{\ifx\@docdate\@empty\else\@docdate\fi}
\fancyhead[R]{\ifx\@docauthor\@empty\else\@docauthor\fi}
\headheight 15pt

\def\doclabel#1{\gdef\@doclabel{#1}}
\doclabel{Use {\tt\textbackslash doclabel\{MY LABEL\}}.}
\def\docdate#1{\gdef\@docdate{#1}}
\docdate{Use {\tt\textbackslash docdate\{MY DATE\}}.}
\def\docauthor#1{\gdef\@docauthor{#1}}
\docauthor{Use {\tt\textbackslash docauthor\{MY NAME\}}.}
\makeatother

%% General formatting parameters
\parindent 0pt
\parskip 12pt plus 1pt

% Shortcuts for blackboard bold number sets (reals, integers, etc.)
\newcommand{\Reals}{\ensuremath{\mathbb R}}
\newcommand{\Nats}{\ensuremath{\mathbb N}}
\newcommand{\Ints}{\ensuremath{\mathbb Z}}
\newcommand{\Rats}{\ensuremath{\mathbb Q}}
\newcommand{\Cplx}{\ensuremath{\mathbb C}}
%% Some equivalents that some people may prefer.
\let\RR\Reals
\let\NN\Nats
\let\II\Ints
\let\CC\Cplx

%%%%%%%%%%%%%%%%%%%%%%%%%%%%%%%%%%%%%%%%%%%%%%%%%%%%%%%%%%%%%%%%%%%%%%%%%%%%%%%%%%%%%%%
%%%%%%%%%%%%%%%%%%%%%%%%%%%%%%%%%%%%%%%%%%%%%%%%%%%%%%%%%%%%%%%%%%%%%%%%%%%%%%%%%%%%%%%
% 
% The main document start here.

% The following commands set up the material that appears in the header.
\doclabel{Math 426: Homework 4}
\docauthor{Stefano Fochesatto}
\docdate{September 23, 2020}

\newcommand{\vv}{\mathbf{v}}
\begin{document}


















\begin{exercise}{Chapter 4: 2 (c)}
For full credit you must write your own version of the secant method.
Your function should have the signature

\begin{verbatim} 
function [r,hist] = secant(f,x0,x1,ftol,xtol,Nmax)

end
\end{verbatim}

The input values are
\begin{itemize}
\item \texttt{f}, the function to find a root of.
\item \texttt{x0}, \texttt{x1} the first two iteration values.
\item \texttt{ftol}, the tolerance for stopping based of the value of \texttt{f}
\item \texttt{xtol}, the tolerance for stopping based on changes in \texttt{x}
\item \texttt{Nmax}, the maximum number of iterations
\end{itemize}

Your function should exit with an error if more than \texttt{Nmax} iterations
are used.  It should return whenever $|f(x)|<f_{\text{tol}}$ or $|x_n-x_{n-1}|<x_{\text{tol}}$.

The return values should be \texttt{r}, the estimate of the root's position,
and \texttt{hist}, a list of all estimates starting with \texttt{x0} and \texttt{x1} and ending with the final estimate \texttt{r}.

Test that your function works by finding three different ways to call it
so that iteration stops for each of the three possible reasons.

To answer the problem in the textbook, you will want to call your function with 
$x_{\rm tol}=0$ to ensure that only the $f_{\rm tol}$ condition is used
to stop the iteration.
\end{exercise}


\textbf{Code:}
\lstinputlisting{hw4secant.m}


\solution Testing all the exit conditions on the function,
\begin{equation*}
  f(x) = x^2 - 4.
\end{equation*}
\textbf{Console:}
\lstinputlisting{hw4}
Running our secant method with the function,
\begin{equation*}
  f(x) = (5-x)e^x - 5,
\end{equation*}
where, $x_0 = 4$,$x_1 = 5$ and our $f(x)$ tolerance set to $10^{-8}$.  \\

\textbf{Console:}
\lstinputlisting{hw4.1}
Using lemma 4.17 from the textbook we can approximate the magnitude of the next few errors,
\begin{equation*}
  e_{k+1}\approx 1.62(e_k)(e_{k-1})
\end{equation*}
\textbf{Console:}
\lstinputlisting{hw4.3}
Thus we can see that after two more iteration through the secant method we will be inside an an $|f(x)| = 10^-16$ 




\begin{exercise}{Chapter 4: 3} Newton's Method can be used to compute reciprocals, without division. To compute $1/R$,
  let $f(x) = x^{-1} - R$ so that $f(x) = 0$ when $x = 1/R$. Write down the Newton iteration for this problem, and compute the 
  first few Newton iterates for approximating $1/3$, starting at $x_0 = .5$, and not using any division. What happens if ou start with
  $x_0 = 1$? For positive $R$, use theory of fixed point iteration to determine an interval about $1/R$ from which Newton's method will converge to 
  $1/R$.


  \solution Consider the Newton iteration, 
  \begin{align*}
    x_{n+1} &= x_n - \dfrac{f(x_n)}{f'(x_n)},\\
    x_{n+1} &= x_n + (x_n^{-1} - R)(x^2),\\
    x_{n+1} &= 2x_n - Rx_n^2.
  \end{align*}
Solving the first few terms for the iteration where $R = 3$ and $x_0 = .5$,
\begin{align*}
  x_0 &= .5,\\
  x_1 &= .25,\\
  x_2 &= .3125,\\
  x_3 &= .33203125.
\end{align*}
Solving the first few terms for the iteration where $R = 3$ and $x_0 = 1$,
\begin{align*}
  x_0 &= 1,\\
  x_1 &= -1,\\
  x_2 &= -5,\\
  x_3 &= -185.
\end{align*}
It's likely that the sequence will diverge to $-\infty$.\\

Using Theorem 4.5.1 we can solve for the interval about $1/R$ for which Newton's method will converge. Note that we can 
write our Newton iteration as a fixed point problem where,
\begin{equation*}
  x_{n+1} = \phi(x_n).
\end{equation*}
Also note that $\phi(x) \in C^1$. Now consider,
\begin{equation*}
  |\phi'(x)|<1.
\end{equation*}
Expanding the inequality and substituting $\phi'(x)$,
\begin{align*}
  -1 < &\phi'(x) < 1\\
  -1 < & 2 - 2Rx < 1\\
  -3 < & - 2Rx < -1\\
  \dfrac{1}{2R} < &x < \dfrac{3}{2R}.
\end{align*}
Thus if $x_0$ is inside the interval $(\dfrac{1}{2R},\dfrac{3}{2R})$ the fixed point iteration $\phi(x)$ will converge. 

\end{exercise}






\begin{exercise}{Chapter 4: 8} In using the secant method to find a root, $x_0 = 2$, $x_1 = -1$ $x_2 = -2$ with 
  $f(x_1) = 4$ and $f(x_2) = 3$. What is $f(x_0)$?
\solution
Consider the iteration function for the secant method,
\begin{equation*}
  x_{n+1} = x_n - \dfrac{f(x_n)(x_n - x_{n-1})}{f(x_n) - f(x_{n-1})}
\end{equation*}
Through substitution we get,
\begin{align*}
  -2 &= -1 - \dfrac{-12}{4 - f(x_{0})}\\
  1 &=\dfrac{-12}{4 - f(x_{0})}\\
  4 - f(x_{0}) &=-12\\
  f(x_{0}) &=16.
\end{align*}






\end{exercise}

\begin{exercise}{Chapter 4: 12} Let the function,
  \begin{equation*}
    \phi(x) = \dfrac{x^2+4}{5} 
  \end{equation*}
  \begin{enumerate}
    \item[\textbf{a.}] Find the fixed points of $\phi(x)$.\\
     
    \solution By definition we know that a fixed point is where $\phi(x)=x$, therefore solving for the fixed points,
    \begin{align*}
    x &= \dfrac{x^2+4}{5}\\
    5x &= x^2+4\\
    0 &= x^2 - 5x + 4\\
    0 &= (x-1)(x-4)
    \end{align*}
    Thus the fixed points are $x = 4$ and $x = 1$.
    \vspace{.25in} 
    \item[\textbf{b.}] Would the fixed point iteration, $x_{k+1} = \phi(x_k)$, converge to a fixed point in the interval $[0,2]$ 
    for all initial guesses $x_0 \in [0,2]$?\\
    
    \solution  Using Theorem 4.5.1 we can solve for the interval for which $x_{k+1} = \phi(x_k)$ will converge. Note that $\phi(x) \in C^1$. Now consider,
    \begin{equation*}
      |\phi'(x)|<1.
    \end{equation*}
    Expanding the inequality and substituting $\phi'(x)$,
    \begin{align*}
      -1 < &\phi'(x) < 1\\
      -1 < &\dfrac{2}{5}x< 1\\
      -\dfrac{5}{2} < &x < \dfrac{5}{2}\\
    \end{align*}
    Thus if $x_0$ is inside the interval $(-\dfrac{2}{5},\dfrac{2}{5})$ the fixed point iteration $\phi(x)$ will converge.
  \end{enumerate}
\end{exercise}















\begin{exercise}{Supplemental 1}
Suppose $f(x)$ is a differentiable function 
on $\mathbb{R}$ and $|f'(x)|\le 1/2$ for
all real numbers.  Show that
\[
|f(x)-f(y)|\le \frac{1}{2} |x-y|
\]
for all $x,y\in \mathbb{R}$.

Hint: Use Taylor's theorem, the zeroth order version, AKA the Mean Value Theorem.  Apply it centered at some point $x$ and then see what the theorem says about $f(y)$.

For context, look at equation (4.22) of the text, which defines a \textbf{contraction}.  You are showing that if $|f'(x)|\le 1/2$ for all $x$ then
$f$ is a contraction.  This is interesting because Theorem 4.5.2 says
that every contraction has a unique fixed point, and if you perform fixed
point iteration on the contraction then the iterates will converge to the fixed point.  You can think of Theorem 4.5.2 as a generalization of Theorem 4.5.1 (you can prove Theorem 4.5.1 directly from the more difficult Theorem 4.5.2).
\end{exercise}

\solution
Suppose that  $f(x)$ is a differentiable function 
on $\mathbb{R}$ and $|f'(x)|\le 1/2$ for
all real numbers. Now consider the first zeroth order Taylor polynomial, 
\begin{equation*}
  f(x) = f(y) + f'(\xi)(x - y).
\end{equation*}
Where $\xi,y \in \Reals$. Now through some algebra we get that,
\begin{align*}
  f(x) &= f(y) + f'(\xi)(x - y),\\
  f(x) - f(y) &= f'(\xi)(x - y),\\
  |f(x) - f(y)| &= |f'(\xi)(x - y)|,\\
  |f(x) - f(y)| &= |f'(\xi)||(x - y)|.
\end{align*}
Note that since $|f'(x)| \le \frac{1}{2}$ then we know that,
\begin{equation*}
  |f(x) - f(y)| =  \frac{1}{2}|(x - y)|,
\end{equation*}
for all values $x,y\in \mathbb{R}$.














\begin{exercise}{Chapter 4: 13} Consider the $a = y - \epsilon sin(y)$, where $0 < \epsilon < 1$ is given and
  $a \in [0,\pi]$ is given. Write this in the form of a fixed point problem for the unknown solution of $y$
  and show that it has a unique solution. [Hint: You will want to use Theorem 4.5.2!]
\end{exercise} Consider the equation,

\solution Consider the fixed point equation,
\begin{equation*}
  \phi(y) = a + \epsilon sin(y)
\end{equation*}
Now consider the derivative of $\phi(y)$
\begin{equation*}
  \phi'(y) =\epsilon cos(y).
\end{equation*}
Note that since the function $cos(y) \in [-1,1]$ and $0 < \epsilon < 1$ we know that for all $\epsilon$,
\begin{equation*}
  \phi'(y) < 1.
\end{equation*}
Now consider the zeroth order taylor polynomial for the function $\phi$,
\begin{align*}
  \phi(x) &= \phi(y) + \phi'(\xi)(x - y),\\
  \phi(x) - \phi(y) &= \phi'(\xi)(x - y),\\
  |\phi(x) - \phi(y)| &= |\phi'(\xi)(x - y)|,\\
  |\phi(x) - \phi(y)| &= |\phi'(\xi)||(x - y)|.
\end{align*}
Recall that since,
\begin{equation*}
  \phi'(y) < 1
\end{equation*}
is true for all values of $\epsilon$ we know, by Theorem 4.5.2 that $\phi$ is a contraction and therefore it has a unique fixed point
$x_*$ and the corresponding iteration function,
\begin{equation*}
  x_{k+1} = \phi(x_k),
\end{equation*}
converges to $x_*$ for any $x_0$.


\begin{exercise}{Chapter 4: 14} If you enter a number into handheld calculator and repeatedly press the cosine
  button, what number will approximately will appear? Provide a proof.  
\end{exercise}

\solution




We can model the action of repeatedly pressing the cosine button as an iteration function,
\begin{equation}
  x_{k+1} = cos(x_k),
\end{equation}
Where the corresponding fixed point equation,
\begin{equation*}
  \phi(x) = cos(x).
\end{equation*}
Note, that any fixed point for this equation must be contained where $x \in [-1,1]$. 
Considering again the zeroth order taylor polynomial for the function $\phi$,
\begin{align*}
  cos(x) &= cos(y) + -sin(\xi)(x - y),\\
  cos(x) - cos(y) &= -sin(\xi)(x - y),\\
  |cos(x) - cos(y)| &= |-sin(\xi)(x - y)|,\\
  |cos(x) - cos(y)| &= |-sin(\xi)||(x - y)|.
\end{align*}
Since $|-sin(\xi)|>1$ for all values $\xi \in [-1,1]$ then, by Theorem 4.5.2 that $\phi$ is a contraction and therefore it has a unique fixed point.
Approximating that fixed point using matlab,

\textbf{Console:}
\lstinputlisting{hw4.6}

\end{document}