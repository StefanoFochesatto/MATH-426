%%%%%%%%%%%%%%%%%%%%%%%%%%%%%%%%%%%%%%%%%%%%%%%%%%%%%%%%%%%%%%%%%%%%%%%%%%%%%%%%%%%%%%%
%%%%%%%%%%%%%%%%%%%%%%%%%%%%%%%%%%%%%%%%%%%%%%%%%%%%%%%%%%%%%%%%%%%%%%%%%%%%%%%%%%%%%%%
% 
% This top part of the document is called the 'preamble'.  Modify it with caution!
%
% The real document starts below where it says 'The main document starts here'.

\documentclass[12pt]{article}

\usepackage{amssymb,amsmath,amsthm}
\usepackage[top=1in, bottom=1in, left=1.25in, right=1.25in]{geometry}
\usepackage{fancyhdr}
\usepackage{graphicx}
\usepackage{enumerate}
\usepackage{verbatim}
\usepackage{listings}

% Comment the following line to use TeX's default font of Computer Modern.
\usepackage{times,txfonts}

\newtheoremstyle{homework}% name of the style to be used
  {18pt}% measure of space to leave above the theorem. E.g.: 3pt
  {12pt}% measure of space to leave below the theorem. E.g.: 3pt
  {}% name of font to use in the body of the theorem
  {}% measure of space to indent
  {\bfseries}% name of head font
  {:}% punctuation between head and body
  {2ex}% space after theorem head; " " = normal interword space
  {}% Manually specify head
\theoremstyle{homework} 

% Set up an Exercise environment and a Solution label.
\newtheorem*{exercisecore}{\@currentlabel}
\newenvironment{exercise}[1]
{\def\@currentlabel{#1}\exercisecore}
{\endexercisecore}

\newcommand{\localhead}[1]{\par\smallskip\noindent\textbf{#1}\nobreak\\}%
\newcommand\solution{\localhead{Solution:}}



% \newcommand{includematlab}[1]{\verbatiminput{#1}}

%%%%%%%%%%%%%%%%%%%%%%%%%%%%%%%%%%%%%%%%%%%%%%%%%%%%%%%%%%%%%%%%%%%%%%%%
%
% Stuff for getting the name/document date/title across the header
\makeatletter
\RequirePackage{fancyhdr}
\pagestyle{fancy}
\fancyfoot[C]{\ifnum \value{page} > 1\relax\thepage\fi}
\fancyhead[L]{\ifx\@doclabel\@empty\else\@doclabel\fi}
\fancyhead[C]{\ifx\@docdate\@empty\else\@docdate\fi}
\fancyhead[R]{\ifx\@docauthor\@empty\else\@docauthor\fi}
\headheight 15pt

\def\doclabel#1{\gdef\@doclabel{#1}}
\doclabel{Use {\tt\textbackslash doclabel\{MY LABEL\}}.}
\def\docdate#1{\gdef\@docdate{#1}}
\docdate{Use {\tt\textbackslash docdate\{MY DATE\}}.}
\def\docauthor#1{\gdef\@docauthor{#1}}
\docauthor{Use {\tt\textbackslash docauthor\{MY NAME\}}.}
\makeatother

%% General formatting parameters
\parindent 0pt
\parskip 12pt plus 1pt

% Shortcuts for blackboard bold number sets (reals, integers, etc.)
\newcommand{\Reals}{\ensuremath{\mathbb R}}
\newcommand{\Nats}{\ensuremath{\mathbb N}}
\newcommand{\Ints}{\ensuremath{\mathbb Z}}
\newcommand{\Rats}{\ensuremath{\mathbb Q}}
\newcommand{\Cplx}{\ensuremath{\mathbb C}}
%% Some equivalents that some people may prefer.
\let\RR\Reals
\let\NN\Nats
\let\II\Ints
\let\CC\Cplx

%%%%%%%%%%%%%%%%%%%%%%%%%%%%%%%%%%%%%%%%%%%%%%%%%%%%%%%%%%%%%%%%%%%%%%%%%%%%%%%%%%%%%%%
%%%%%%%%%%%%%%%%%%%%%%%%%%%%%%%%%%%%%%%%%%%%%%%%%%%%%%%%%%%%%%%%%%%%%%%%%%%%%%%%%%%%%%%
% 
% The main document start here.

% The following commands set up the material that appears in the header.
\doclabel{Math 426: Homework 6}
\docauthor{Stefano Fochesatto}
\docdate{\today}

\newcommand{\vv}{\mathbf{v}}
\begin{document}

\begin{exercise}{Problem 7.1} Write the following matrix in the form $LU$, where $L$ is a unit lower triangular matrix and $U$ is an upper triangular matrix,
	\begin{equation*}
		\begin{pmatrix}4&-1&-1\\ -1&4&-1\\ -1&-1&4\end{pmatrix}
	\end{equation*}
	Step one of gaussian elimination by reducing the terms of the first column,
	\begin{equation*}
		L_1A = \begin{pmatrix}1&0&0\\ \frac{1}{4}&1&0\\ \frac{1}{4}&0&1\end{pmatrix} \begin{pmatrix}4&-1&-1\\ -1&4&-1\\ -1&-1&4\end{pmatrix} = \begin{pmatrix}4&-1&-1\\ 0&\frac{15}{4}&-\frac{5}{4}\\ 0&-\frac{5}{4}&\frac{14}{4}\end{pmatrix}.
	\end{equation*}
	
	Step two we reduce the terms in the second column and we get $L$ and $U$,
	\begin{equation*}
		L_2L_1A =  \begin{pmatrix}1&0&0\\ 0&1&0\\ 0&\frac{1}{3}&1\end{pmatrix}  \begin{pmatrix}4&-1&-1\\ 0&\frac{15}{4}&-\frac{5}{4}\\ 0&-\frac{5}{4}&\frac{14}{4}\end{pmatrix} = \begin{pmatrix}4&-1&-1\\ 0&\frac{15}{4}&-\frac{5}{4}\\ 0&0&\frac{10}{3}\end{pmatrix}.
	\end{equation*}
		Therefore we know that,
		\begin{equation*}
			L = L_1^{-1}L_2^{-1} = \begin{pmatrix}1&0&0\\ -\frac{1}{4}&1&0\\ -\frac{1}{4}&-\frac{1}{3}&1\end{pmatrix}
		\end{equation*}
		\begin{equation*}
			U = \begin{pmatrix}4&-1&-1\\ 0&\frac{15}{4}&-\frac{5}{4}\\ 0&0&\frac{10}{3}\end{pmatrix}
		\end{equation*}
		Now consider the following equation,
		\begin{equation*}
			A = LL^{T}.
		\end{equation*}
		Through matrix multiplication we see that $LL^{T}$ results in the following symmetric matrix.
		\begin{equation*}
		LL^{T} ={\begin{pmatrix}L_{11}^{2}&   &   ({\text{symmetric}})\\L_{21}L_{11}&   L_{21}^{2}+L_{22}^{2}&   \\L_{31}L_{11}&   L_{31}L_{21}+L_{32}L_{22}&   L_{31}^{2}+L_{32}^{2}+L_{33}^{2}\end{pmatrix}}.
		\end{equation*}
		Plugging in our values from $A$ and solving for the $L_{ij}$ terms we get,
		\begin{equation*}
			L =  \begin{pmatrix}2&0&0\\ -\frac{1}{2}&\frac{\sqrt{15}}{2}&0\\ -\frac{1}{2}&-\frac{\sqrt{15}}{6}&\frac{\sqrt{30}}{2} \end{pmatrix}
		\end{equation*}
		Therefore writing $A$ in terms of its $LL^{T}$ factorization,
		\begin{equation*}
			A =  \begin{pmatrix}2&0&0\\ -\frac{1}{2}&\frac{\sqrt{15}}{2}&0\\ -\frac{1}{2}&-\frac{\sqrt{15}}{6}&\frac{\sqrt{30}}{2} \end{pmatrix} \begin{pmatrix}2&-\frac{1}{2}&-\frac{1}{2}\\ 0&\frac{\sqrt{15}}{2}&-\frac{\sqrt{15}}{6}\\ 0&0&\frac{\sqrt{30}}{2}\end{pmatrix}
		\end{equation*}
		We can also double check our algebra using Matlab,\\
		\textbf{Console:}
		\begin{center}
			\lstinputlisting{algebra.txt}
		\end{center}








	\end{exercise}
\vspace{.5in}








\begin{exercise}{Problem 7.2} Write a function $usolve$, analogous to the function $lsolve$ in section 7.2.2. to solve
	an upper triangular system, $Ux = y$. \\

	\textbf{Code:}
	\begin{center}
		\lstinputlisting{usolve.m}
	\end{center}

	Testing the code with a couple of problems,\\

	\textbf{Console:}
	\begin{center}
		\lstinputlisting{usolvetest.txt}
	\end{center}




\end{exercise}


\vspace*{1in}










\begin{exercise} {Problem 7.3 [Modified]} \strut

\begin{itemize}
	\item By hand, solve the following linear system exactly,
	\begin{equation*}
		A = \begin{pmatrix}
			10^{-16} && 1\\
			1 && 1
		\end{pmatrix},
		b = \begin{pmatrix}
			2\\
			3
		\end{pmatrix}.
	\end{equation*}
	Write your answer in the form that it is clear what the approximate values of $x_1$ and $x_2$ are.\\

	\solution
	First we will perform gaussian elimination on the augmented matrix $Ab$.
	\begin{equation*}
		\begin{pmatrix}
			10^{-16} && 1 &&| 2\\
			1 && 1 &&| 3
		\end{pmatrix}
		\to
		\begin{pmatrix}
			10^{-16} && 1 &&| 2\\
			0 && 1 - 10^{-16} &&| 3 - 2*10^{-16}
		\end{pmatrix}
	\end{equation*}
Solving the system of linear equation, we get,
\begin{equation*}
	x_2 = \dfrac{3 - 2\times 10^{-16}}{1 - \times 10^{-16}} \approx \dfrac{- 2\times 10^{-16}}{-\times 10^{-16}} \approx 2
\end{equation*}
\begin{equation*}
x_1 = (2 - \dfrac{3 - 2\times 10^{-16}}{1 - \times 10^{-16}})10^{16} \approx (2 - \dfrac{-2\times 10^{-16}}{-\times 10^{-16}})10^{16} \approx (2 - 2) 10^{16} \approx 0
\end{equation*}

\vspace*{.25in}



	\item Write a Matlab function {\tt LUNoPivot} that 
	takes as input a square matrix and returns two matrices $L$ and $U$,
	lower and upper triangular matrices such that $L$ has 1's
	on the diagonal and such that $A=LU$.  Do not pivot (i.e., do not perform row interchanges).  You can use the code on page 140
	of your text as a starting point. You should test your code on the 
	$3\times 3$ matrix presented in class today; the matrix $A$ from
	page 135.  That is, verify that indeed $LU=A$.

	Note that the code on page 140 is being sneaky.  Rather than building two matrices, it builds just one. Since $L$ always has 1s on the diagonal, it only has interesting entries below the diagonal.  And since $U$ is all zeros below the diagonal, there's space there to store the entries of $L$!  This is an important space saving technique when the matrices involved are large: no need to go around working with extra matrices that are half zeros and use up twice the needed storage.  But for the purposes of this exercise and clarity,
	we'll return $L$ and $U$ separately.\\

	\textbf{Code:}
	\begin{center}
		\lstinputlisting{LUNoPivot.m}
	\end{center}
	Testing our code with the 3x3 matrix presented in class.\\

	\textbf{Console:}
	\begin{center}
		\lstinputlisting{LUNoPivotTest.txt}
	\end{center}
\vspace{.25in}

	\item Use {\tt lsolve} from the text (page 140) and write matlab that solves the linear system 
	$Ax = b$. Compare the answer to this code to the one you determined by hand.

	\solution We simply write a function that calls our prior three functions appropriately to produce a solution. Consider,\\

	\textbf{Code:}

	
	\begin{center}
		\lstinputlisting{NoPivotSolve.m}
	\end{center}

	Using our code to solve $Ax = b$\\

	\textbf{Console:}
	\begin{center}
		\lstinputlisting{NoPivotSolveTest.txt}
	\end{center}

\end{itemize}
\end{exercise}

\vspace{1in}



\begin{exercise}{Problem 7.4}  The matrix $P$ in this
problem is called a permutation matrix.  We'll discuss this more when we
cover pivoting.  But you can still work on this problem. The
first step to solving $Ax=b$ in this context is to multiply the equation
by $P$.  Notice that all $P$ does in rearrange the entries of $b$: that's
why it's called a permutation matrix!\\

\solution Consider the following,
\begin{align*}
	Ax &= b\\
	PAx &= Pb.
\end{align*}
Note that we have the $LU$ factorization for the matrix $PA$ therefore all we must do is solve for $Pb$ and
continue with solving the system by $LU$ factorization as normal. Through matrix multiplication,
\begin{equation*}
	Pb = \begin{pmatrix}
		0 &&0 &&1\\ 
		1 &&0&& 0\\
		0 &&1 &&0
	\end{pmatrix}
	\begin{pmatrix}
		2\\
		10\\
		-12
	\end{pmatrix}
	 = 
	 \begin{pmatrix}
		-12\\
		2\\
		10
	\end{pmatrix}.
\end{equation*}
Then we continue by solving the system $L\hat{b} =Pb$,
\begin{equation*}
	\begin{pmatrix}
		1 &&0 &&0\\ 
		\frac{1}{2} &&1&& 0\\
		\frac{1}{3} &&\frac{1}{4}&&1
	\end{pmatrix}
	\begin{pmatrix}
		\hat{b}_1\\
		\hat{b}_2\\
		\hat{b}_3
	\end{pmatrix} = 
	\begin{pmatrix}
		-12\\
		2\\
		10
	\end{pmatrix}.
\end{equation*}
Doing so we get, 
\begin{equation*}
	\hat{b} = 	\begin{pmatrix}
		-12\\
		12\\
		8
	\end{pmatrix}.
\end{equation*}
Finally we solve the system by solving $Ux = \hat{b}$
\begin{equation*}
	\begin{pmatrix}
		2 &&3 &&1\\ 
		0 &&1&& 2\\
		0 &&0 &&2
	\end{pmatrix}	
	\begin{pmatrix}
		  x_1\\
		  x_2\\
		  x_3
	\end{pmatrix}
	 = 
	 \begin{pmatrix}
		-12\\
		12\\
		8
	\end{pmatrix}.
\end{equation*}
Therefore the solution to $Ax = b$ is,
\begin{equation*}
	x = \begin{pmatrix}-14\\ 4\\ 4\end{pmatrix}
\end{equation*}
\end{exercise}
\vspace{1in}









\begin{exercise}{Problem 7.6} How many operations are requresd to computre the following,
	\begin{enumerate}
		\item Compute the sum of two $n$-vectors?\\
		
		\solution When we add or subtract vectors we add or subtract the correspoding socmponents therefore there are
		$2n$ operations when computing the sum of two $n$-vectors. 
		\vspace{.25in}



		\item Compute the product of an $m$ by $n$ matrix with an $n$-vector?\\
		
		\solution To compute the product of a matrix and a vector, we take the dot product of each row vector from the matrix with the vector. 
		Note that the dot product of two $n$-vectors has $n$ multiplications and $n-1$ additions, and in this case there are $m$ dot products. Thus
		the total number of computations is, $m(2n-1)$. 




		\vspace{.25in}



		\item Solve an $n$ by $n$ upper triangular linear system $Ux = y$?\\
		
		\solution Recall that we did this exercise in class. Consider the form of $x_1$,
		\begin{equation*}
			x_1 = \dfrac{y_1 - a_2x_2 - a_3x_3 - a_4x_4 ... - a_nx_n}{a_1}
		\end{equation*}
		where $a_i = U(1,i)$. Now note that for each $x_n$ computation there are $n-1$ multiplications, $n-1$ subtractions, and 1 division so $2n-1$ operations total. 
		To solve the whole system we must compute all $x_n$ therefore the total number of computations is,
		\begin{equation*}
			\sum_{i = 1}^n 2i - 1 = n^2.
		\end{equation*}  
		Note that the algebra for this computation was done in class and on the worksheet. 
	\end{enumerate}
\end{exercise}

\end{document}
