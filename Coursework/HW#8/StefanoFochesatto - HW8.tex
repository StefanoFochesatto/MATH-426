%%%%%%%%%%%%%%%%%%%%%%%%%%%%%%%%%%%%%%%%%%%%%%%%%%%%%%%%%%%%%%%%%%%%%%%%%%%%%%%%%%%%%%%
%%%%%%%%%%%%%%%%%%%%%%%%%%%%%%%%%%%%%%%%%%%%%%%%%%%%%%%%%%%%%%%%%%%%%%%%%%%%%%%%%%%%%%%
% 
% This top part of the document is called the 'preamble'.  Modify it with caution!
%
% The real document starts below where it says 'The main document starts here'.

\documentclass[12pt]{article}

\usepackage{amssymb,amsmath,amsthm}
\usepackage[top=1in, bottom=1in, left=1.25in, right=1.25in]{geometry}
\usepackage{fancyhdr}
\usepackage{graphicx}
\usepackage{enumerate}
\usepackage{verbatim}
\usepackage{listings}


% Comment the following line to use TeX's default font of Computer Modern.
\usepackage{times,txfonts}

\newtheoremstyle{homework}% name of the style to be used
  {18pt}% measure of space to leave above the theorem. E.g.: 3pt
  {12pt}% measure of space to leave below the theorem. E.g.: 3pt
  {}% name of font to use in the body of the theorem
  {}% measure of space to indent
  {\bfseries}% name of head font
  {:}% punctuation between head and body
  {2ex}% space after theorem head; " " = normal interword space
  {}% Manually specify head
\theoremstyle{homework} 

% Set up an Exercise environment and a Solution label.
\newtheorem*{exercisecore}{\@currentlabel}
\newenvironment{exercise}[1]
{\def\@currentlabel{#1}\exercisecore}
{\endexercisecore}

\newcommand{\localhead}[1]{\par\smallskip\noindent\textbf{#1}\nobreak\\}%
\newcommand\solution{\localhead{Solution:}}



% \newcommand{includematlab}[1]{\verbatiminput{#1}}

%%%%%%%%%%%%%%%%%%%%%%%%%%%%%%%%%%%%%%%%%%%%%%%%%%%%%%%%%%%%%%%%%%%%%%%%
%
% Stuff for getting the name/document date/title across the header
\makeatletter
\RequirePackage{fancyhdr}
\pagestyle{fancy}
\fancyfoot[C]{\ifnum \value{page} > 1\relax\thepage\fi}
\fancyhead[L]{\ifx\@doclabel\@empty\else\@doclabel\fi}
\fancyhead[C]{\ifx\@docdate\@empty\else\@docdate\fi}
\fancyhead[R]{\ifx\@docauthor\@empty\else\@docauthor\fi}
\headheight 15pt

\def\doclabel#1{\gdef\@doclabel{#1}}
\doclabel{Use {\tt\textbackslash doclabel\{MY LABEL\}}.}
\def\docdate#1{\gdef\@docdate{#1}}
\docdate{Use {\tt\textbackslash docdate\{MY DATE\}}.}
\def\docauthor#1{\gdef\@docauthor{#1}}
\docauthor{Use {\tt\textbackslash docauthor\{MY NAME\}}.}
\makeatother

%% General formatting parameters
\parindent 0pt
\parskip 12pt plus 1pt

% Shortcuts for blackboard bold number sets (reals, integers, etc.)
\newcommand{\Reals}{\ensuremath{\mathbb R}}
\newcommand{\Nats}{\ensuremath{\mathbb N}}
\newcommand{\Ints}{\ensuremath{\mathbb Z}}
\newcommand{\Rats}{\ensuremath{\mathbb Q}}
\newcommand{\Cplx}{\ensuremath{\mathbb C}}
%% Some equivalents that some people may prefer.
\let\RR\Reals
\let\NN\Nats
\let\II\Ints
\let\CC\Cplx

%%%%%%%%%%%%%%%%%%%%%%%%%%%%%%%%%%%%%%%%%%%%%%%%%%%%%%%%%%%%%%%%%%%%%%%%%%%%%%%%%%%%%%%
%%%%%%%%%%%%%%%%%%%%%%%%%%%%%%%%%%%%%%%%%%%%%%%%%%%%%%%%%%%%%%%%%%%%%%%%%%%%%%%%%%%%%%%
% 
% The main document start here.

% The following commands set up the material that appears in the header.
\doclabel{Math 426: Homework 8}
\docauthor{Stefano Fochesatto}
\docdate{October 21, 2020}

\begin{document}
\begin{exercise}{Supplemental 1}
Finish your code from question 19 of the worksheet on implementing
LU decomposition with partial pivoting.  Then use this code together
with the course page \texttt{lsolve} and your own \texttt{usolve}
to solve $A\mathbf{x}=\mathbf{b}$ where
\[
A=\begin{pmatrix} 9 & 3 & 2 & 0 &7 \\
7 & 6 & 9 & 6 & 4\\
2 & 7 & 7 & 8 & 2 \\
0 & 9 & 7 & 2 & 2 \\
7 & 3 & 6 & 4 & 3
\end{pmatrix}
\]
and $b=[35,58,53,37,39]^T$.  For the record, the true solution is $x=[0,1,2,3,4]^T$.
\end{exercise}
\solution The following is a modified LUPivot Function that takes an augmented matrix in the form of 
the matrix $A$ and the vector $b$ and returns the pivoted $LU$ factorization as well as the pivoted vector $b'$ where
$Pb = b'$. The Usolve and LSolve function remain unchanged from last week.\\

\textbf{Code:}\\
\begin{center}
  \lstinputlisting{LUPivot.m}
\end{center}

Finally we can write a function that solves a matrix equation using $LU$ factorization
with partial pivoting,\\

\textbf{Code:}\\
\begin{center}
  \lstinputlisting{PivotSolve.m}
\end{center}




\textbf{Console}
\begin{center}
  \lstinputlisting{PivotTest.txt}
\end{center}










\begin{exercise}{Exercise 7.8} Compute the $2-norm$, the $1-norm$ and the $\infty-norm$ of,
  \begin{equation*}
    v = \begin{pmatrix}
      4\\
      5\\
      -6
    \end{pmatrix}
  \end{equation*}
\end{exercise}
\solution By definition we can compute all vector norms easily,
\begin{equation*}
  [v]_2 = \sqrt{4^2+5^2+(-6)^2} = \sqrt{77}
\end{equation*}
\begin{equation*}
  [v]_1 = |4| + |5| + |-6| = 15
\end{equation*}
\begin{equation*}
  [v]_{\infty} = max\{|v|\} = 6 
\end{equation*}

\end{document}