%%%%%%%%%%%%%%%%%%%%%%%%%%%%%%%%%%%%%%%%%%%%%%%%%%%%%%%%%%%%%%%%%%%%%%%%%%%%%%%%%%%%%%%
%%%%%%%%%%%%%%%%%%%%%%%%%%%%%%%%%%%%%%%%%%%%%%%%%%%%%%%%%%%%%%%%%%%%%%%%%%%%%%%%%%%%%%%
% 
% This top part of the document is called the 'preamble'.  Modify it with caution!
%
% The real document starts below where it says 'The main document starts here'.

\documentclass[12pt]{article}

\usepackage{amssymb,amsmath,amsthm}
\usepackage[top=1in, bottom=1in, left=1.25in, right=1.25in]{geometry}
\usepackage{fancyhdr}
\usepackage{graphicx}
\usepackage{enumerate}
\usepackage{verbatim}

% Comment the following line to use TeX's default font of Computer Modern.
\usepackage{times,txfonts}

\newtheoremstyle{homework}% name of the style to be used
  {18pt}% measure of space to leave above the theorem. E.g.: 3pt
  {12pt}% measure of space to leave below the theorem. E.g.: 3pt
  {}% name of font to use in the body of the theorem
  {}% measure of space to indent
  {\bfseries}% name of head font
  {:}% punctuation between head and body
  {2ex}% space after theorem head; " " = normal interword space
  {}% Manually specify head
\theoremstyle{homework} 

% Set up an Exercise environment and a Solution label.
\newtheorem*{exercisecore}{\@currentlabel}
\newenvironment{exercise}[1]
{\def\@currentlabel{#1}\exercisecore}
{\endexercisecore}

\newcommand{\localhead}[1]{\par\smallskip\noindent\textbf{#1}\nobreak\\}%
\newcommand\solution{\localhead{Solution:}}



% \newcommand{includematlab}[1]{\verbatiminput{#1}}

%%%%%%%%%%%%%%%%%%%%%%%%%%%%%%%%%%%%%%%%%%%%%%%%%%%%%%%%%%%%%%%%%%%%%%%%
%
% Stuff for getting the name/document date/title across the header
\makeatletter
\RequirePackage{fancyhdr}
\pagestyle{fancy}
\fancyfoot[C]{\ifnum \value{page} > 1\relax\thepage\fi}
\fancyhead[L]{\ifx\@doclabel\@empty\else\@doclabel\fi}
\fancyhead[C]{\ifx\@docdate\@empty\else\@docdate\fi}
\fancyhead[R]{\ifx\@docauthor\@empty\else\@docauthor\fi}
\headheight 15pt

\def\doclabel#1{\gdef\@doclabel{#1}}
\doclabel{Use {\tt\textbackslash doclabel\{MY LABEL\}}.}
\def\docdate#1{\gdef\@docdate{#1}}
\docdate{Use {\tt\textbackslash docdate\{MY DATE\}}.}
\def\docauthor#1{\gdef\@docauthor{#1}}
\docauthor{Use {\tt\textbackslash docauthor\{MY NAME\}}.}
\makeatother

%% General formatting parameters
\parindent 0pt
\parskip 12pt plus 1pt

\def\vx{\mathbf x}
\def\vb{\mathbf b}

% Shortcuts for blackboard bold number sets (reals, integers, etc.)
\newcommand{\Reals}{\ensuremath{\mathbb R}}
\newcommand{\Nats}{\ensuremath{\mathbb N}}
\newcommand{\Ints}{\ensuremath{\mathbb Z}}
\newcommand{\Rats}{\ensuremath{\mathbb Q}}
\newcommand{\Cplx}{\ensuremath{\mathbb C}}
%% Some equivalents that some people may prefer.
\let\RR\Reals
\let\NN\Nats
\let\II\Ints
\let\CC\Cplx

%%%%%%%%%%%%%%%%%%%%%%%%%%%%%%%%%%%%%%%%%%%%%%%%%%%%%%%%%%%%%%%%%%%%%%%%%%%%%%%%%%%%%%%
%%%%%%%%%%%%%%%%%%%%%%%%%%%%%%%%%%%%%%%%%%%%%%%%%%%%%%%%%%%%%%%%%%%%%%%%%%%%%%%%%%%%%%%
% 
% The main document start here.

% The following commands set up the material that appears in the header.

%%%%%%%%%%%%%%%%%%%%%%%%%%%%%%%%%%%%%%%%%%%%%%%%%%%%%%%%%%%%%%%%%%%%%%%%%%
\doclabel{Math 426: Homework 12}
\docauthor{Stefano Fochesatto}
\docdate{November 18, 2020}

\begin{document}


\begin{exercise}{Text 8.10} Let $f$ be the function satisfying $f(0) = 1$, $f(1) = 2$, and $f(2) = 0$. A quadratic spline 
  interpolant $r(x)$ is defines as a piecewise quadratic that interpolates $f$ at the nodes $(x_0 = 0, x_1 = 1, x_2 = 2)$
  and whose first derivative is continuous throughout the interval. Find the quadratic spline interpolant of $f$ that also
  satisfies $r'(0) = 0$.\\

\solution Note that since $r(x)$ is a piecewise quadratic interpolant on nodes $(x_0 = 0, x_1 = 1, x_2 = 2)$, it must be of the form,
\begin{equation*}
   r(x) = 
   \begin{cases} 
    a_0x^2+b_0x+c_0 & 0\leq x\leq 1 \\
    a_1x^2+b_1x+c_1 & 1< x\leq 2 
 \end{cases}
\end{equation*}
Note that our interpolant has the property that $r(x) = f(x)$ and $r'(x) = f'(x)$ on our sample points there fore we get the following 
system of equations,
\begin{equation*}
  a_0(0)^2+b_0(0)+c_0(1) = 1,
\end{equation*}
\begin{equation*}
  a_0(1)^2+b_0(1)+c_0(1) = 1,
\end{equation*}
\begin{equation*}
  a_02(0)+b_0(1)+c_0(0) = 0.
\end{equation*}
Solving we get that $a_0 = 1,b_0 = 0,c_0 = 1$, which gives us the function,
\begin{equation*}
  r(x) = \begin{cases} 
    x^2 + 1 & 0 \leq x \leq 1
 \end{cases}
\end{equation*}
Differentiating we get $r'(1) = 2$. Setting up a new system to solve $a_1, b_1, c_1$.
\begin{equation*}
  a_1(1)^2+b_1(1)+c_1(1) = 2,
\end{equation*}
\begin{equation*}
  a_1(2)^2+b_1(2)+c_1(1) = 0,
\end{equation*}
\begin{equation*}
  a_12(1)+b_1(1)+c_1(0) = 2.
\end{equation*}
Solving we get that $a_1 = -4, b_1 = 10, c_1 = -4$. Therefore we get,
\begin{equation*}
  r(x) = 
  \begin{cases} 
    x^2 + 1 & 0\leq x\leq 1 \\
   -4x^2 + 10x - 4 & 1< x\leq 2 
\end{cases}
\end{equation*}
\end{exercise}
\vspace{.5in}





















\begin{exercise}{Text 8.12} Show that the following function is a natural cubic spline through the
  points $(0,1)$, $(1,1)$, $(2,0)$, and $(3,10)$:
  \begin{equation*}
    s(x) = 
    \begin{cases} 
     1 + x - x^3 & 0\leq x < 1 \\
     1 - 2(x-1) - 3(x - 1)^2 + 4(x - 1)^3 & 1 \leq x < 2\\
     4(x - 2) + 9(x - 2)^2 - 3(x - 2)^3 & 2 \leq x \leq 3 
  \end{cases}
 \end{equation*}
\end{exercise}














\begin{exercise}{Text 8.13}
\end{exercise}

\begin{exercise}{Text 8.14}
\end{exercise}

\begin{exercise}{Text 10.1}
\end{exercise}

\begin{exercise}{Text 10.2}
\end{exercise}


\end{document}